% latex uft-8
\documentclass[uplatex, a4paper, 10pt, oneside, twocolumn]{jsarticle}
%
\usepackage[dvipdfmx]{graphicx}
\usepackage{here}
\usepackage{amsmath}
\usepackage{amssymb}
\usepackage{comment}
\usepackage[labelsep=space]{caption}
\usepackage[subrefformat=parens, labelsep=space]{subcaption}
\usepackage{siunitx}
\usepackage{bm}

\captionsetup{compatibility=false}
%J\usepackage{amsmath、 amssymb}
%J\usepackage{bm}
%\usepackage{graphicx}
%\usepackage{ascmac}
%
\setlength{\textwidth}{\fullwidth}
\setlength{\textheight}{40\baselineskip}
\addtolength{\textheight}{\topskip}
\setlength{\voffset}{-0.55in}


\title{心理測定関数のロジスティック回帰}
\author{瀧澤哲}
\date{2020年5月1日}
%
\begin{document}
% START DOCUMENT
%

\maketitle

\section*{解答}
  \subsection*{ステップ1 尤度関数を計算}
    \begin{equation}
      PF(X;M,S) = \frac{1}{1 + \exp \left(\frac{M-X_i}{S}\right)}
    \end{equation}
    \begin{equation}
      \begin{split}
        P &= \prod_{i=0}^n \left[ PF(X_i; M)^{C_i} * \left\{1 - PF(X_i; M)\right\}^{T_i - C_i} \right] \\
          &= \prod_{i=0}^n \left[ \left(\frac{1}{1+\exp \left(\frac{M-X_i}{S}\right)}\right)^{C_i} * \left\{1 - \frac{1}{1 + \exp \left(\frac{M-X_i}{S}\right)}\right\}^{T_i - C_i} \right]
      \end{split}
    \end{equation}

  \subsection*{ステップ2 尤度関数を最大にするMを計算}
    \begin{equation}
      \begin{split}
        \log (P) &= L  \\
          &= \log \left[\prod_{i=0}^n \left[ PF(X_i; M)^{C_i} * \left\{1 - PF(X_i; M)\right\}^{T_i - C_i} \right] \right] \\
          &= \sum_{i=0}^n \left\{ -T_i \log \left(1+\exp\left(\frac{M-X_i}{S}\right)\right) + (T_i - C_i) \left(\frac{M-X_i}{S}\right) \right\}
      \end{split}
    \end{equation}
    

% end document
\end{document}
